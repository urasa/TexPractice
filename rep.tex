\documentclass[twocolumn]{jsarticle}
\begin{document}

\title{タイトル}
\author{著者名}
\maketitle

\section{はじめに}

この文書は、ごく基本的なレポートや論文の例を示すものです。
実際にこのソースを入力してタイプセット(コンパイル)し、
タイトル、著者名、本文,見出し,箇条書きが
どのように表示されるかを確認してみましょう。

\section{見出し}

この文書の先頭にはタイトル,書写名,日付が出力されています。
特定の日付を指定することもできます。

\section{箇条書き}

以下は箇条書きのれいです。これは番号を降らない箇条書きです。

\begin{itemize}
  \item りぼん
  \item なかよし
\end{itemize}

これは番号をふる箇条書きです。

\begin{enumerate}
  \item 富士
  \item 鷹
  \item なすび
\end{enumerate}

\section{おわりに}

これは一段組の例ですが,二段組にすることもできます。

解説文をよんで,このソースをいろいろと変更してみましょう。

\end{document}


